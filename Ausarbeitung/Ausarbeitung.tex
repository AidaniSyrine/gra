% Diese Zeile bitte -nicht- aendern.
\documentclass[course=erap]{aspdoc}

%%%%%%%%%%%%%%%%%%%%%%%%%%%%%%%%%
%% TODO: Ersetzen Sie in den folgenden Zeilen die entsprechenden -Texte-
%% mit den richtigen Werten.
\newcommand{\theGroup}{IhreGruppennummer} % Beispiel: 42
\newcommand{\theNumber}{IhreProjektnummer} % Beispiel: A123
\author{Vorname1 Nachname1 \and Vorname2 Nachname2 \and Vorname3 Nachname3}
\date{AktuellesSemester} % Beispiel: Wintersemester 2019/20
%%%%%%%%%%%%%%%%%%%%%%%%%%%%%%%%%

% Diese Zeile bitte -nicht- aendern.
\title{Gruppe \theGroup{} -- Abgabe zu Aufgabe \theNumber}

\begin{document}
\maketitle

\section{Einleitung}
Der Schwerpunkt dieses Projekts liegt auf der Konvertierung farbiger Bilder in Graustufen und der darauf folgenden Durchführung einer Tonwertkorrektur. 
Als Eingabe lesen wir zuerst 24bpp PPM (P6) Pixelbilder. Ein Farbpixel wird als Vektor mit den Farbkanälen Rot R, Grün G und Blau B definiert.
Durch die Berechnung eines gewichteten Durchschnitts D, wandeln wir zunächst die Bilder in Graustufen um. Die entstandenen Graustufenbilder unterziehen wir dem Tonwertkorrektur. 
Hierbei dienen die nutzerspezifizierte Werte für Schwarz, Mitten und Weiß (ES, AS), (EM, AM) und (EW, AW) als Stützpunkte für eine Interpolationsfunktion, die ermöglicht eine reibungslose Anpassung der Graustufenwerte zwischen den nutzerspezifizierte Stützpunkten. 
Zuletzt erstellen wir eine neue Datei im 8bpp PGM (P5) Format und dort hineinschreiben wir die berechneten Werte nach dem Tonwertkorrektur.
In den kommenden Abschnitten werden wir die korrektheit der Implementierung und der gewählten Ansätze prüfen und das Laufzeitverhalten und die Performanz evaluieren.

\section{Lösungsansatz}
\subsection{Graustufen Konvertierung}

\subsubsection{Intuitiver Ansatz: Durchschnitt}
die folgende Umrechnungsformel ist einfach der Durchschnitt der RGB-Komponenten:
D = (R + G + B) / 3 
Die Durchschnittsmethode ist ebenfalls problematisch, da sie jedem Komponenten den gleichen Gewicht zuweist.

\subsubsection{Luminanz Ansatz}
Die Luminanz Y dient als Maß für die Helligkeit von Bildpunkten, entsprechend der Wahrnehmung durch das menschliche Auge. Da das Auge besonders empfindlich für die Farbe Grün ist und weniger empfindlich für Blau, wird eine Gewichtung der RGB-Komponenten benötigt, um die Farbwahrnehmung des menschlichen Auges zu berücksichtigen. [1]
In unserem Kontext, bei der Arbeit mit einem PPM-Bild, das der ITU-R-Empfehlung BT.709 entspricht [2], wird die folgende Umrechnungsformel gemäß Rec. 709 für die Berechnung der CIE-Luminanz aus den linearen RGB-Komponenten verwendet:
Y = 0.2125R + 0.7154G + 0.0721B  [3]
Die ITU-R Empfehlung BT.709 definiert auch das gamma-codierte Luma Y': eine gewichtete Summe der nichtlinearen (nach Gamma-Korrektur) RGB-Komponenten, wobei gilt:
Y' = 0.2126R' + 0.7152G' + 0.0722B' [4]
Die Verwendung der Luminanz als Maß bietet eine genauere Darstellung der tatsächlichen Helligkeit und eignet sich daher besser für die 8-Bit-Kodierung von Graustufen. Man muss jedoch das Bild zuvor in einen linearen RGB-Farbraum umwandeln, um den gewichteten Durchschnitt auf die linearen RGB-Komponenten anwenden zu können.
Alternativ kann jede der drei Farbkomponenten auf das berechnete Luma Y' gesetzt werden. Dies ermöglicht eine einfachere Berechnung und ist in der Praxis oft ausreichend für Graustufenbilder.
D = Y' = 0.2126R' + 0.7152G' + 0.0722B' (  0.2126 + 0.7152 + 0.0722 = 1)

\subsubsection{Grün-Kanal Ansatz}
Das menschliche Auge ist am empfindlichsten gegen Grün, daher ist die Verwendung von Grün eine schnellere , aber ungenauere  Alternative zur Umwandlung von RGB in Luminanz.
D = G (a = 0 ; b = 1 ; c = 0)


% TODO: Je nach Aufgabenstellung einen der Begriffe wählen
\section{Korrektheit/Genauigkeit}


\section{Performanzanalyse}


\section{Zusammenfassung und Ausblick}

% TODO: Fuegen Sie Ihre Quellen der Datei Ausarbeitung.bib hinzu
% Referenzieren Sie diese dann mit \cite{}.
% Beispiel: CR2 ist ein Register der x86-Architektur~\cite{intel2017man}.
\bibliographystyle{plain}
\bibliography{Ausarbeitung}{}

\end{document}
